
\documentclass[a4paper,12pt]{article}
\usepackage[utf8]{inputenc}
\usepackage{amsmath}
\usepackage{geometry}
\geometry{margin=2.5cm}
\usepackage{lmodern}
\usepackage{graphicx}
\usepackage{amsfonts}
\usepackage{hyperref}

\title{Cheat Sheet: Standardavvik og Gjennomsnitt}
\author{}
\date{}

\begin{document}

\maketitle

\section*{Introduksjon}
Dette dokumentet gir en rask oversikt over hvordan man regner ut gjennomsnitt ($\mu$) og standardavvik ($\sigma$ og $s$), med eksempler og forklaringer.

\section*{1. Gjennomsnitt ($\mu$)}
\textbf{Formel for populasjonsgjennomsnitt:}
\begin{equation}
\mu = \frac{1}{n} \sum_{i=1}^{n} x_i
\end{equation}

\textbf{Eksempel:}\\
Tall: 5, 7, 9, 4, 10\\
\begin{align*}
\mu &= \frac{5 + 7 + 9 + 4 + 10}{5} = \frac{35}{5} = 7
\end{align*}

\textbf{Forklaring:}\\
Gjennomsnittet $\mu$ forteller hva verdiene i datasettet i snitt ligger på. Det er et mål på sentraltendens og brukes ofte i statistikk.

\section*{2. Standardavvik for populasjon ($\sigma$)}
\textbf{Formel:}
\begin{equation}
\sigma = \sqrt{ \frac{1}{n} \sum_{i=1}^{n} (x_i - \mu)^2 }
\end{equation}

\textbf{Eksempel:}\\
Tall: 2, 4, 4, 4, 5, 5, 7, 9\\
Gjennomsnitt $\mu = 5$\\
\begin{align*}
\sigma &= \sqrt{ \frac{1}{8} \left[(2-5)^2 + (4-5)^2 + \dots + (9-5)^2 \right] } \\
       &= \sqrt{4} = 2
\end{align*}

\textbf{Forklaring:}\\
Standardavviket $\sigma$ måler spredningen i hele populasjonen, altså hvor mye dataene varierer rundt gjennomsnittet $\mu$.

\section*{3. Standardavvik for utvalg ($s$)}
\textbf{Formel:}
\begin{equation}
s = \sqrt{ \frac{1}{n-1} \sum_{i=1}^{n} (x_i - \bar{x})^2 }
\end{equation}

\textbf{Eksempel:}\\
Tall: 4, 8, 6, 5, 3\\
Gjennomsnitt $\bar{x} = 5.2$\\
\begin{align*}
s &= \sqrt{ \frac{1}{4} \left[(4-5.2)^2 + (8-5.2)^2 + \dots + (3-5.2)^2 \right] } \\
  &\approx \sqrt{3.7} \approx 1.92
\end{align*}

\textbf{Forklaring:}\\
Utvalgsstandardavvik $s$ brukes når vi kun har et utvalg og ikke hele populasjonen. $n-1$ brukes i nevneren som en korreksjon (Bessels korreksjon) for å få et mer nøyaktig estimat.

\end{document}
